\documentclass[./main.tex]{subfiles}

\begin{document}
\section{Fondamenti di elettrostatica}
\subsection{Frigoelettricità e cariche elettriche}
La teoria dell'elettromagnetismo, sviluppatasi soprattutto nel XIX secolo, prende avvio da un fatto già noto nella Grecia classica:
\begin{fact}
Un pezzo di ambra\footnote{in greco \texteta \textlambda \textepsilon \textchi \texttau \textrho \textomikron \textnu , da cui il termine "elettrico" stesso.} strofinato con un panno di lana inizia ad attrarre corpi molto leggeri.
\end{fact}
Questo fenomeno è noto come \emph{frigoelettricità}. Un'analisi più approfondita, compiuta variando sia il materiale delle bacchette strofinate che quello del panno, rivela alcune caratteristiche dell'interazione tra bacchette:
\begin{fact}
La natura della forza che agisce su due bacchette strofinate con panni e avvicinate varia a seconda della combinazione dei materiali. Non tutti i materiali danno origine a una forza; tra quelli che lo fanno, esistono due specie: il "tipo vetro" ed il "tipo bachelite". Due bacchette della stessa specie si respingono, mentre due di specie opposte si attraggono. In ogni caso, tra una bacchetta ed il panno usato per strofinarla si stabilisce sempre una forza attrattiva.
\end{fact}
Vale la pena chiedersi quale sia la natura di questa forza. Essa infatti è esercitata a distanza; l'unica forza a distanza in meccanica classica è la gravità, ma questa è sempre attrattiva e per di più le masse coinvolte non cambiano apprezzabilmente durante lo strofinio. Sembra proprio di essere davanti a un nuovo tipo di forza. Chiamiamo questa forza \emph{forza elettrica} ed i corpi su cui essa agisce \emph{elettrizzati}.

Per caratterizzare la forza, ipotizziamo una nuova proprietà dei corpi: la \emph{carica elettrica}. Essa può essere positiva o negativa. Per convenzione, la carica di un corpo elettrizzato di tipo vetro è positiva, mentre è negativa quella di un corpo elettrizzato di tipo bachelite. A questo punto, possiamo riscrivere le osservazioni precedenti con la nuova terminologia: due corpi con cariche dello stesso segno si respingono, due con cariche di segno opposto si attraggono, e durante lo strofinio bacchetta e panno acquistano cariche di segno opposto.

L'elettrizzazione agisce in maniera diversa su materiali diversi. In alcuni materiali, detti \emph{isolanti}, essa rimane concentrata vicino alla zona dello strofinio; in altri, detti \emph{conduttori}, essa si diffonde anche in zone lontane. Inoltre, un corpo conduttore si scarica talmente in fretta da risultare impossibile da elettrizzare, se viene tenuto in mano: ciò si spiega col fatto che sia il corpo umano che la superficie terrestre sono conduttori, e dunque la carica tende a passare attraverso il corpo e diffondersi su tutta la Terra. Dato che quest'ultima è ovviamente preponderante rispetto al conduttore, la carica risulta disperdersi. Se invece un conduttore è circondato da isolanti, la carica rimane su di esso, non potendo attraversare gli isolanti. Un corpo elettrizzato disperde comunque la sua carica, anche se lentamente, per via della conduttività dell'aria.

\subsubsection{Elettroscopio}
Un \emph{elettroscopio} è uno strumento che consente di riconoscere la carica relativa di due corpi. Esso è costituito da due fogli sottili di metallo collegati a una bacchetta conduttrice e protetti da un contenitore di vetro chiuso da un tappo isolante, completato da una scala graduata per misurare l'angolo tra i due fogli. Toccando la parte esposta dell'asticella con un corpo carico, infatti, esso comunica parte della sua carica alla bacchetta e, tramite essa, ai foglietti, i quali, avendo cariche concordi, si respingono allontanandosi dalla verticale fino a raggiungere una posizione di equilibrio. Toccando poi la parte esposta con un altro corpo carico, i foglietti si allontanano ulteriormente o si riavvicinano a seconda che questo abbia carica concorde o discorde all'altro.

Il fatto che due oggetti producano la stessa deflessione implica solo che essi abbiano caricato \emph{i foglietti} in maniera uguale, non che le loro cariche siano uguali. Due oggetti vengono invece assunti avere cariche uguali quando, posti alla stessa distanza da un terzo oggetto carico, esercitano su questo la stessa forza; cariche opposte quando esercitano forze uguali in modulo ma di verso opposto.

\subsubsection{Induzione elettrostatica}
Si osserva anche che avvicinando una bacchetta carica a un elettroscopio -- senza toccarlo -- le foglie divergono, riavvicinandosi poi una volta che la bacchetta viene allontanata. Questo fenomeno è dovuto a un processo di separazione di carica, non di sua aggiunta, noto come \emph{induzione elettrostatica}. Si può indurre una carica permanente su un conduttore isolandolo da terra mentre un corpo carico è tenuto vicino ad esso, impedendo così alle cariche di ritornare nella situazione iniziale quando il corpo viene rimosso.

\subsubsection{Natura microscopica della carica elettrica}
Oggi è noto che la materia con cui si interagisce quotidianamente è costituita da atomi, a loro volta costituiti da protoni (carichi positivamente), neutroni (neutri) ed elettroni (carichi negativamente). Solitamente protoni e neutroni sono presenti negli atomi in egual numero; dato che $q_{p} = -q_{e} = -e$, ciò porta la materia ad essere in gran parte elettricamente neutra. Se però gli atomi di un oggetto perdono o guadagnano elettroni, esso si carica positivamente o negativamente. Inoltre, gli elettroni si spostano più o meno facilmente all'interno di materiali diversi.

Quando si strofina una bacchetta, lo sfregamento causa il passaggio di elettroni dalla bacchetta al panno per i materiali che si caricano positivamente come il vetro, nel senso opposto per i materiali che si caricano negativamente come la bachelite. Nei materiali isolanti queste cariche restano dove sono, producendo la forza elettrica osservata; nei conduttori gli elettroni si spostano attraverso la bacchetta e il corpo di chi la sta tenendo in mano per riempire le lacune lasciate negli atomi del vetro o per dissiparsi nel terreno, rispettivamente. Se però il conduttore è a contatto solo con isolanti, le cariche non possono entrarvi né uscirne (in seguito si dimostrerà che esse si distribuiscono però solo sulla superficie del corpo). In ogni caso, anche se la carica dei singoli corpi varia, \emph{la carica del sistema si conserva} ed è indipendente dal sistema di riferimento\footnote{A differenza della massa, che a velocità relativistiche \emph{dipende} dal sistema di riferimento.}.

Dato che la carica elettrica è dovuta a quella delle particelle elementari, essa è quantizzata, ovvero i corpi possono variare la loro carica solo in incrementi di $\pm e$. Per via del piccolo valore di questa carica, tuttavia, si può benissimo trattare la carica come una grandezza continua a livelli macroscopici.

\subsection{Legge di Coulomb per cariche puntiformi}
Nel 1785, Charles-Augustin de Coulomb compì varie misure sulla forza elettrica agente su sferette attaccate a una bilancia di torsione (vedi appendice \ref{BT-App:BilTorsione} per il funzionamento di quest'ultima), variando sistematicamente tutti i parametri\footnote{Le cariche si possono variare grazie al fatto che due conduttori sferici a contatto si ripartiscono la carica totale in maniera direttamente proporzionale ai raggi.} dell'esperimento. Egli ottenne i seguenti risultati, validi per sferette quasi puntiformi:
\begin{fact}[Esperimento di Coulomb]
La forza elettrostatica tra due sferette cariche può essere sia attrattiva che repulsiva, a seconda che queste cariche siano discordi o concordi. Essa è diretta lungo la congiungente ed è proporzionale alle cariche e inversamente proporzionale al quadrato della distanza tra di esse.
\end{fact}
In simboli, si ha che 
\begin{mdframed}
\begin{equation} \label{eq:Coulomb}
\mathbf{F} = k \dfrac{Qq}{R^2}\boldsymbol{\mathbf{\hat{R}}}
\end{equation}
dove $\mathbf{F}$ è la forza elettrica che il corpo con carica $Q$ esercita su quello con carica $q$, $\mathbf{R} = R \boldsymbol{\mathbf{\hat{R}}}$ è il vettore spostamento dal corpo carico $Q$ a quello carico $q$ e $k$ è una costante di proporzionalità.
\end{mdframed}
Si noti che questa legge è in accordo con il principio di azione e reazione: un corpo carico esercita su un altro una forza uguale e contraria a quella che il secondo esercita sul primo. Inoltre, la forza elettrostatica è centrale e proporzionale al quadrato della distanza, come quella gravitazionale.

I principali limiti alla precisione dell'esperimento di Coulomb derivano dal fatto che, perché l'approssimazione di carica puntiforme sia ragionevole, la distanza tra le sferette deve essere molto maggiore del raggio, ma un aumento della distanza porta a una riduzione della forza, che diventa quindi più difficile da misurare, e una riduzione del raggio delle sferette porta a una più rapida dispersione della carica nell'aria che le circonda. Bisogna inoltre considerare il fatto che questa dispersione è comunque sempre presente, e infine schermare il sistema dall'influenza di corpi carichi circostanti. L'accuratezza della legge è supportata, più che da misure dirette, dalla validità dei risultati ricavati da essa.

\subsubsection{Valore di \texorpdfstring{$k$}{k} ed unità di misura della carica elettrica}
Riguardo alla costante di proporzionalità $k$ si possono prendere due possibili strade, usate in due diversi sistemi di unità di misura. La prima, usata dal sistema CGS\footnote{Centimetri, Grammi, Secondi.}, è porre a priori $k=1$; la seconda, adottata dal sistema MKS\footnote{Metri, Kilogrammi, Secondi; è il Sistema Internazionale.}, è definire un'unità di misura per la carica e calcolare $k$ dai dati sperimentali.
\paragraph{Sistema CGS}
Avendo posto a priori $k=1$, ed essendo dunque $[k]=[1]$, si può ricavare la dimensione della carica elettrica dalla legge di Coulomb:
\begin{gather}
[F] = [M] [L] [T]^{-2} = [Q]^{2} [L]^{-2} \\
[Q] = [M]^{1/2} [L]^{3/2} [T]^{-1}
\end{gather}
La dimensione carica risulta essere definita in termini meccanici (massa, lunghezza e tempo) ma poco intuitivi (a cosa corrisponde la radice di una massa?). Per questo motivo, il SI preferisce adottare un'altra strada.
\paragraph{Sistema MKS}
Si definisce un'unità di misura della carica separata da quelle meccaniche, il \emph{coulomb} $\mathrm{C}$, corrispondente alla carica trasportata in $1$ secondo da una corrente di $1$ ampere, la quale a sua volta è definita in modo che la carica di un elettrone risulti essere $e = -1.6021766208\ \mathrm{C}$. A questo punto si può misurare il valore di $k$, che risulta essere 
\begin{equation}
k := \dfrac{1}{4 \pi \ezero}=8.9875 \cdot 10^{9}\ \mathrm{Nm^{2}/C^{2}},
\end{equation}
dove per avere formule più semplici in seguito si è espresso $k$ in termini di una costante detta \emph{costante dielettrica o permettività del vuoto} (l'aria presente intorno alle sferette nell'esperimento di Coulomb può essere approssimata al vuoto),
\begin{equation}
\ezero = 8.8542 \cdot 10^{-12}\ \mathrm{C^{2}/Nm^{2}}
\end{equation}

\subsection{Campo elettrostatico}
La forza a cui è soggetta una carica puntiforme $q$ sotto l'azione di un'altra $Q$ è data dalla legge di Coulomb (eq. \ref{eq:Coulomb}); essa si può però esprimere in una forma alternativa come
\begin{mdframed}
\begin{equation}
\mathbf{F} = q \left[ \dfrac{1}{4 \pi \ezero} \dfrac{Q}{r^{2}} \rhat \right] := q \mathbf{E}(Q,\mathbf{r})
\end{equation}
$\mathbf{E}$ è un vettore che, per un punto fissato individuato da $\mathbf{r}$, non dipende dalla carica su cui agisce la forza, ma solo da quella che la esercita; viceversa, per ogni punto dello spazio è possibile definire un valore di $\mathbf{E}$. Esso costituisce quindi un campo vettoriale, denominato \emph{campo elettrostatico}, e la carica $Q$ è detta \emph{sorgente}. Al momento esso sembra un formalismo matematico, ma si vedrà che esso ha un'esistenza fisica in quanto aspetto del campo elettrico, che è un'entità esistente a prescindere dall'esistenza delle cariche, ma genera forze interagendo con esse.
\end{mdframed}

Le unità di misura del (modulo del) campo elettrico sono
\begin{equation}
[E] = [F] [Q]^{-1} =\ \mathrm{N/C\ nel\ SI}
\end{equation}
Come d'altra parte tutti i campi vettoriali, $\mathbf{E}$ può essere rappresentato con \emph{linee di campo}, ovvero linee ad esso tangenti e concordi in ogni loro punto e con densità proporzionale al suo modulo in quel punto. Analiticamente, le linee di campo sono definite come percorsi a partire da punti arbitrari composti da spostamenti infinitesimi $\mathbf{\dx{l}}(x,y,z) = (\dx{x}, \dx{y}, \dx{z})$ tali che
\begin{equation}
\dfrac{\dx{x}}{E_{x}(x,y,z)} = \dfrac{\dx{y}}{E_{y}(x,y,z)} = \dfrac{\dx{z}}{E_{z}(x,y,z)}
\end{equation}

Il campo elettrostatico segue il \emph{principio di sovrapposizione}, ovvero il campo prodotto da un insieme di cariche è la somma (vettoriale) dei campi che ognuna di esse produrrebbe in assenza delle altre. Infatti,
\begin{equation}
\mathbf{F}_{T} = \sum_{i} \mathbf{F}_{i} = \sum_{i} q \mathbf{E}_{i} = q \sum_{i} \mathbf{E}_{i} \Rightarrow \mathbf{E}_{T} := \dfrac{\mathbf{F}_{T}}{q} = \sum_{i} \mathbf{E}_{i}
\end{equation}
Nei casi concreti, vale a dire con oggetti carichi non puntiformi, un oggetto carico \emph{può} cambiare il campo elettrico in maniera non obbediente al principio di sovrapposizione, ma questo non è dovuto ad un'interazione diretta bensì ad una redistribuzione di cariche tramite processi come l'induzione elettrostatica o la polarizzazione. Questi diminuiscono di intensità al ridursi della carica $q$, per cui è più preciso definire il campo elettrico come 
\begin{equation}
\mathbf{E} = \lim_{q_{0}\rightarrow 0} \dfrac{\mathbf{F}}{q_{0}}
\end{equation}

\subsection{Calcolo del campo elettrostatico tramite il principio di sovrapposizione}
Il principio di sovrapposizione consente, in linea di principio, di calcolare il campo elettrostatico dovuto a un qualsiasi sistema, finito o infinito, discreto o continuo, di cariche, riconducendosi alla legge di Coulomb.

\subsubsection{Insieme discreto di cariche puntiformi} \label{subsec:CoulombDiscr}
Si prenda un sistema di riferimento con origine in $O$; siano $\mathbf{r}_{1},...,\mathbf{r}_{n}$ i vettori posizione delle $n$ cariche $Q_{1},..., Q_{n}$ e $\mathbf{r}_{P}$ il vettore posizione del punto $P$. Siano per brevità $\Delta \mathbf{r}_{i} := \mathbf{r}_{P} - \mathbf{r}_{i}$. Allora, per il principio di sovrapposizione, si ha che
\begin{mdframed}
\begin{equation}
\mathbf{E}(P) = \dfrac{1}{4 \pi \ezero} \sum _{i=1} ^{n} \dfrac{Q_{i}}{\Delta r_{i} ^{2}} \Delta \rhat_{i}
\end{equation}
\end{mdframed}

\subsubsection{Distribuzione continua di carica} \label{subsec:CoulombCont}
La legge di Coulomb si applica anche al campo generato da cariche infinitesime $\dx{q}$, nella forma
\begin{equation} \label{eq:InfinitesimalCoulomb}
\dx{\mathbf{E}} = \dfrac{1}{4 \pi \ezero} \dfrac{\dx{q}}{r^{2}} \rhat
\end{equation}
Per poter calcolare il campo complessivo, si definiscono le \emph{densità di carica}, rispettivamente:
\begin{itemize}
	\item \emph{volumetrica} in un volume infinitesimo $\dx{\tau}$ con posizione $\mathbf{r}$ e avente carica infinitesima $\dx{q}$:
	\begin{equation}
 	\rho (\mathbf{r}) := \lim _{\Delta \tau \rightarrow 0} \dfrac{\Delta q}{\Delta \tau} = \dfrac{\dx{q}}{\dx{\tau}}
 	\end{equation}
	\item \emph{superficiale} su una superficie infinitesima $ \dx{s}$ con posizione $\mathbf{r}$ e avente carica infinitesima $\dx{q}$:
	\begin{equation}
 	\sigma (\mathbf{r}) := \lim _{\Delta s \rightarrow 0} \dfrac{\Delta q}{\Delta s} = \dfrac{\dx{q}}{\dx{s}}
 	\end{equation}
 	\item \emph{lineare} su un segmento infinitesimo $\dx{l}$ con posizione $\mathbf{r}$ e avente carica infinitesima $\dx{q}$:
	\begin{equation}
 	\lambda (\mathbf{r}) := \lim _{\Delta l \rightarrow 0} \dfrac{\Delta q}{\Delta l} = \dfrac{\dx{q}}{\dx{l}}
 	\end{equation}
\end{itemize}

$\dx{\tau}$, $\dx{s}$, $\dx{l}$ sono all'atto pratico \emph{differenziali fisici}, ovvero sono piccoli abbastanza da poter essere considerati puntiformi ma non abbastanza da dover tenere conto delle proprietà microscopiche della materia. Se per esempio si considerasse un $\dx{\tau}$ delle dimensioni di un nucleo, il valore di $\rho$ varierebbe fortemente da un punto all'altro, a seconda della presenza o meno di protoni e/o elettroni in esso.
Risulta ovvio dalle definizioni che a seconda del tipo di distribuzione $\dx{q}$ è pari a $\rho \dx{\tau}$, $\sigma \dx{s}$ o $\lambda \dx{l}$; dunque, sostituendo nella \ref{eq:InfinitesimalCoulomb} e integrando, si ha
\begin{mdframed}
\begin{gather}
\mathbf{E}(P) = \dfrac{1}{4 \pi \ezero} \int _{\tau} \dfrac{\rho (\mathbf{r})}{\Delta r^{2}} \Delta \rhat \ \dx{\tau} \\
\mathbf{E}(P) = \dfrac{1}{4 \pi \ezero} \int _{s} \dfrac{\sigma (\mathbf{r})}{\Delta r^{2}} \Delta \rhat \ \dx{s} \\
\mathbf{E}(P) = \dfrac{1}{4 \pi \ezero} \int _{l} \dfrac{\lambda (\mathbf{r})}{\Delta r^{2}} \Delta \rhat \ \dx{l}
\end{gather}
\end{mdframed}

Le formule esposte in questo paragrafo e nel precedente sono del tutto generali, ma spesso difficilmente applicabili. In situazioni di particolare semplicità o simmetria del sistema, da esse si possono ricavare formule caratteristiche più interessanti.

\subsubsection{Dipolo elettrico (sull'asse)} \label{subsubsec:DipAsse}
Un dipolo elettrico è costituito da una coppia di cariche puntiformi opposte $\pm q$ separate da una distanza $d$. Scegliamo un sistema di riferimento con le due cariche $\pm q$ sull'asse $z$, rispettivamente in $(0,0,\pm \frac{d}{2})$. Siano $\mathbf{r}_{\pm} = r \mathbf{\hat{r}_{\pm}}$ i vettori uscenti dalle due cariche ed entranti in $P=(0,y,0)$, sull'asse $y$. Il campo elettrostatico in $P$ è dato, secondo il principio di sovrapposizione, da
\begin{align}
\mathbf{E}(P)	&= \dfrac{1}{4 \pi \ezero} \dfrac{q}{r^2} \left(\mathbf{\hat{r}_{+}}+\mathbf{\hat{r}_{-}} \right) \\
				&= \dfrac{1}{4 \pi \ezero} \dfrac{q}{r^2} \left(\dfrac{\mathbf{r}_{+}}{r}+\dfrac{\mathbf{r}_{-}}{r} \right) \\
				&= \dfrac{1}{4 \pi \ezero} \dfrac{q}{r^2} \left(\dfrac{y \jhat + \frac{d}{2} \khat}{r}+\dfrac{y \jhat - \frac{d}{2} \khat}{r} \right) \\
				&= - \dfrac{1}{4 \pi \ezero} \dfrac{q}{r^{3}} d \khat				
\end{align}
Il campo è diretto lungo l'asse $z$, parallelamente al dipolo, dalla carica positiva a quella negativa. Notare che, per la simmetria del problema, questo risultato si applica (ovviamente cambiando la direzione dell'asse $y$) a tutto il piano ortogonale al dipolo e passante per il suo centro. Di solito si esprime il campo come
\begin{mdframed}
\begin{equation}
\mathbf{E}(P)	= - \dfrac{1}{4 \pi \ezero} \dfrac{1}{r^3} \mathbf{P}
\end{equation}
dove
\begin{equation}
\mathbf{P} = qd \khat
\end{equation}
è una grandezza costante su tutto il piano, detta \emph{momento di dipolo elettrico} (NB: $P$ è il punto, $\mathbf{P}$ è il momento di dipolo).
\end{mdframed}

\subsubsection{Distribuzione lineare infinita uniforme} \label{subsubsec:DistrLinUnifSovrapp}
Questo tipo di distribuzione si può pensare come un filo $\gamma$ infinito e uniformemente carico. Scegliamo un sistema di riferimento in cui questo filo coincide con l'asse $z$ e calcoliamo il compo elettrico in un punto $P=(0,y,0)$. Consideriamo coppie di cariche infinitesime $\dx{q}$, poste in $(0,0,\pm z)$, ovvero a distanza $r$ ed angolo $\pm \theta$ da $P$. Il campo infinitesimo risultante da ognuna di queste coppie ha componente $x$ sicuramente nulla, poiché $\mathbf{r}$ è sul piano $y$-$z$; inoltre, esso ha anche componente $z$ nulla, dato che il problema è simmetrico per riflessione rispetto al piano $x$-$y$. Si ha dunque che questo campo infinitesimo ha solo componente $y$ non nulla, e dato che questo ragionamento può essere ripetuto per ogni coppia di cariche, si arriva a un campo risultante
\begin{align}
\mathbf{E}(P)	&= \left[ \int_{\gamma} \dx{E_{y}} \right] \jhat \\
				&= \left[ \int_{\gamma} \dx{\mathbf{E}} \cos \theta \right] \jhat \\
				&= \left[ \int_{- \infty}^{+ \infty} \dfrac{1}{4 \pi \ezero} \dfrac{\lambda \dx{z}}{r^{2}} \cos \theta \right] \jhat \label{line:InfLinearDistr:dz}\\ 
				&= \dfrac{\lambda}{4 \pi \ezero} \left[ \int_{-\frac{\pi}{2}}^{\frac{\pi}{2}} \dfrac{\frac{y}{\cos^{2} \theta} \dx{\theta}}{\left(\frac{y}{\cos \theta}\right)^2} \cos \theta \right] \jhat  \label{line:InfLinearDistr:dtheta}\\ 
				&= \dfrac{\lambda}{4 \pi \ezero} \left[ \sin \theta \right]_{\frac{\pi}{2}}^{\frac{\pi}{2}} \\
				&= \dfrac{1}{2 \pi \ezero} \dfrac{\lambda}{y}
\end{align}
dove per passare dalla \ref{line:InfLinearDistr:dz} alla \ref{line:InfLinearDistr:dtheta} si sono usate opportunamente le relazioni $y = r \cos \theta$ e $z = y \tan \theta$. Come nel paragrafo precedente, variando la direzione dell'asse $y$ questo risultato può essere applicato a tutto il piano, in cui il campo è dunque radiale centrato sul filo e con modulo inversamente proporzionale alla distanza da esso. Si usa quindi esprimerlo nella seguente forma:
\begin{mdframed}
\begin{equation} \label{eq:InfLinearDistrField}
\mathbf{E}(P) = \dfrac{1}{2 \pi \ezero} \dfrac{\lambda}{r} \rhat 
\end{equation}
dove $\mathbf{r} = r \rhat$ è il vettore distanza dal filo al punto.
\end{mdframed}

\subsection{Potenziale elettrostatico}
\subsubsection{Conservatività, energia potenziale, potenziale}
Il campo elettrostatico è un campo conservativo\footnote{Lo sono tutte le forze centrali, tra cui anche la gravità.}. Infatti, indicando con $L$ il lavoro compiuto dalla forza elettrostatica lungo una curva arbitraria $\gamma$ congiungente due punti $A$ e $B$, si ha che
\begin{align}
L 	&= \int_{\gamma} \mathbf{F} \cdot \dx{\boldsymbol{\gamma}} \\
	&= \int_{\gamma} F\ \cos \theta \ \dx{\gamma} \\
	&= \int_{\gamma} F\ \dx{r} \\
	&= \dfrac{Q}{4 \pi \ezero} \int_{\gamma} \dfrac{q}{r^{2}}\ \dx{r} \\
	&= \dfrac{Q}{4 \pi \ezero} \int_{r_{A}}^{r_{B}} \dfrac{q}{r^{2}}\ \dx{r} \\
	&= \dfrac{Qq}{4 \pi \ezero} \left[ - \dfrac{1}{r} \right]_{r_{A}}^{r_{B}} \\
	&= \dfrac{Qq}{4 \pi \ezero} \left( \dfrac{1}{r_{A}} - \dfrac{1}{r_{B}} \right)
\end{align}

Il calcolo si può visualizzare pensando di scomporre uno spostamento arbitrario lungo $\gamma$ in uno radiale di $\dx{r}$, in cui viene compiuto lavoro, a uno tangenziale $r \dx{\theta}$, in cui non viene compito lavoro essendo esso in ogni suo punto perpendicolare al campo. Il lavoro soddisfa dunque il requisito di essere una funzione soltanto dei punti di partenza e di arrivo, e si può quindi definire la funzione scalare \emph{energia potenziale elettrostatica} $U(P)$ tale che
\begin{equation}
U(A) - U(B) := \dfrac{Qq}{4 \pi \ezero} \left( \dfrac{1}{r_{A}} - \dfrac{1}{r_{B}} \right)
\end{equation}
e dunque $L_{AB} = - (U_{B} - U_{A})$. Essendo $U$ definita a partire da una differenza (o equivalentemente, essendo $U(r)$ una primitiva di $F(r)$) essa è definita a meno di una costante:
\begin{mdframed}
\begin{equation}
U(r) = \dfrac{1}{4 \pi \ezero} \dfrac{Qq}{r^{2}} + C
\end{equation}
\end{mdframed}
Spesso si pone $C = 0$, il che equivale a porre $U = 0$ all'infinito, in modo che energia potenziale e interazione tendano a $0$ nelle stesse condizioni (appunto, allontanandosi).

È possibile compiere la stessa operazione direttamente sul campo elettrostatico $\mathbf{E}$. Così facendo (o semplicemente dividendo per $q$) si ha l'espressione del \emph{potenziale elettrostatico} 
\begin{mdframed}
\begin{equation}
V(r) := \dfrac{1}{4 \pi \ezero} \dfrac{Q}{r^{2}} + C
\end{equation}
\end{mdframed}
Esso ha unità di misura $[V] = [\mathbf{E}] [L] = \mathrm{N/C \cdot m := V}$, ovvero \emph{volt}. Spesso anche per esso si pone $C = 0$.

Il campo del potenziale non dipende dalla carica "sonda" $q$ ed è calcolabile a prescindere dalla sua effettiva presenza. Essendo una primitiva, esso è continuo e derivabile. 

In un campo generato da una sola carica, il potenziale dipende solo da $r$, dunque le superfici equipotenziali sono sferiche. Anche nel caso generale, comunque, esse sono perpendicolari alle linee di campo (essendo queste ultime coincidenti col gradiente del potenziale e dovendo essere $\dx{V} = - \bnabla V \cdot \dx{s} = 0$ per uno spostamento sulle equipotenziali) e forniscono quindi una rappresentazione alternativa di quest'ultimo.

Si ricordano le proprietà, tutte equivalenti, di un campo conservativo $\mathbf{E}$:
\begin{itemize}
\item Esiste una funzione $U(P)$ tale che per ogni percorso $AB$ $L_{AB} = U_{A} - U_{B}$ (e nel nostro caso abbiamo definito anche una funzione $V(P)$ tale che $L_{AB} = q[V(A) -V(B)]$).
\item La circuitazione su ogni curva chiusa è nulla.
\item Per ogni punto, il rotore del campo $\bnabla \times \mathbf{E} = 0$. Attenzione: in coordinate polari $\bnabla$ è trasformato.
\end{itemize}

Essendo funzioni primitive, sia energia potenziale che potenziale sono sempre continue e derivabili.

Spesso la circuitazione del campo elettrico è chiamata anche \emph{forza elettromotrice} (fem) e indicata con $\mathcal{E}$. Per il campo elettrostatico sarà sempre $\mathcal{E} = 0$, ma ciò non varrà per il campo elettrico nel caso generale né per il campo magnetico (essi infatti \emph{non} sono conservativi).

Si noti che, anche se dopo aver compiuto un percorso chiuso in un campo elettrostatico l'energia cinetica di una particella carica non può essere cambiata (per via della conservazione dell'energia), può essere cambiata la \emph{direzione} della sua velocità.

La forma differenziale delle equazioni del potenziale è 
\begin{gather}
\dx{V} = - \mathbf{E} \cdot \dx{\mathbf{l}}\\
\mathbf{E} = - \bnabla V
\end{gather}
Queste relazioni risultano particolarmente utili per calcolare il campo elettrico di distribuzioni di cariche, dato che un'integrazione seguita da tre derivazioni è spesso più rapida di tre integrazioni (una per componente). Per un approfondimento su $\bnabla$ e per il suo uso in sistemi di coordinate non cartesiani si veda l'appendice \ref{N-app:nabla}.

\subsection{Potenziale di sistemi e distribuzioni di cariche}
\subsubsection{Principio di sovrapposizione generale}
Il principio di sovrapposizione può essere applicato anche ai campi scalari di energia potenziale e potenziale generati da sistemi di cariche (fisse), grazie alla distributività di sommatorie e integrali. Per un sistema discreto di $n$ cariche, usando la stessa notazione della sezione \ref{subsec:CoulombDiscr}, si ha quindi:
\begin{mdframed}
\begin{gather}
U(P) = \dfrac{q_{P}}{4 \pi \ezero} \sum_{i} \dfrac{Q_{i}}{\Delta r_{i}} + C_1 \\
V(P) = \dfrac{1}{4 \pi \ezero} \sum_{i} \dfrac{Q_{i}}{\Delta r_{i}} + C_2
\end{gather}
\end{mdframed}

E per un sistema continuo, con la notazione della sezione \ref{subsec:CoulombCont}:
\newpage
\begin{mdframed}
\begin{gather}
U(P) = \dfrac{q_{P}}{4 \pi \ezero} \int _{\tau} \dfrac{\rho (\mathbf{r})}{\Delta r} \Delta \rhat \ \dx{\tau} \\
U(P) = \dfrac{q_{P}}{4 \pi \ezero} \int _{s} \dfrac{\sigma (\mathbf{r})}{\Delta r} \Delta \rhat \ \dx{s} \\
U(P) = \dfrac{q_{P}}{4 \pi \ezero} \int _{l} \dfrac{\lambda (\mathbf{r})}{\Delta r} \Delta \rhat \ \dx{l}
\end{gather}
\end{mdframed}

\begin{mdframed}
\begin{gather}
V(P) = \dfrac{1}{4 \pi \ezero} \int _{\tau} \dfrac{\rho (\mathbf{r})}{\Delta r} \Delta \rhat \ \dx{\tau} \\
V(P) = \dfrac{1}{4 \pi \ezero} \int _{s} \dfrac{\sigma (\mathbf{r})}{\Delta r} \Delta \rhat \ \dx{s} \\
V(P) = \dfrac{1}{4 \pi \ezero} \int _{l} \dfrac{\lambda (\mathbf{r})}{\Delta r} \Delta \rhat \ \dx{l}
\end{gather}
\end{mdframed}

\subsubsection{Distribuzione lineare uniforme infinita}
L'energia potenziale di questa distribuzione può essere ricavata facilmente integrando il campo elettrostatico prodotto da essa, fornito dalla eq. \ref{eq:InfLinearDistrField}:
\begin{equation}
U(P_0) - U(P) = \dfrac{\lambda}{2 \pi \ezero} \int_{P_0}^{P} \dfrac{1}{r} = \dfrac{\lambda}{2 \pi \ezero} \ln \left( \dfrac{r}{r_0} \right)
\end{equation}
Questa espressione diverge sia per $P \rightarrow \infty$ che per $P_0 \rightarrow \infty$, rendendo quindi impossibile stabilire un'energia potenziale "rispetto all'infinito" come nel caso di una carica puntiforme. D'altra parte, se si prova a ricavare l'espressione di $U(r)$ tramite il principio di sovrapposizione, si ottiene
\begin{align}
U(y) 	&= \dfrac{1}{4 \pi \ezero} \int \dfrac{\lambda\ \dx{z}}{y} \\
		&= \dfrac{1}{4 \pi \ezero} \int_{- \infty}^{+ \infty} \dfrac{\lambda\ \dx{(r \tan \theta)}}{y} \\
		&= \dfrac{1}{4 \pi \ezero} \int_{- \frac{\pi}{2}}^{\frac{\pi}{2}} \frac{r}{\cos ^2 \theta}{\frac{r}{\cos \theta}}\ \dx{\theta} \\
		&= \dfrac{1}{4 \pi \ezero} \int_{- \frac{\pi}{2}}^{\frac{\pi}{2}} \dfrac{1}{\cos \theta} \dx{\theta} 
\end{align}
il quale diverge. La divergenza è dovuta al fatto che la distribuzione stessa è infinita, e riesce quindi a "contrastare" la diminuzione del campo elettrico con la distanza. Nei casi reali, il problema però non si pone:  dato che $-\frac{\pi}{2} < \theta < \frac{\pi}{2}$, l'integrale è un integrale proprio e porta a un valore finito.

\subsubsection{Distribuzione superficiale uniforme infinita}
Anche in questo caso risulta un potenziale infinito all'infinito, dato che come è facile verificare, essendo $\mathbf{E}$ uniforme risulta 
\begin{equation}
V = - Ez + C
\end{equation}
e anche in questo caso ciò è dovuto al fatto che la distribuzione di carica che genera il campo è infinita. Per inciso, questo infinito è di ordine superiore all'altro, essendo lineare e non logaritmico.

\subsection{Energia di un sistema di cariche} \label{subsec:EnerSist}
Quando si considera una carica (di valore costante $q$) in moto in un campo elettrico esterno la cui origine è ignota o non importante, energia potenziale e potenziale sembrano differire solo per il fattore $q$. Se però si considera un sistema di due o più cariche, mentre il potenziale da esso generato rimane un campo scalare definito in tutto lo spazio, l'energia potenziale elettrostatica deve essere pensata come una proprietà del sistema: essa rappresenta la capacità della forza elettrica di compiere lavoro positivo su di esso portando le cariche all'infinito o, equivalentemente, il lavoro svolto da una forza esterna per costruirlo, portando le cariche dall'infinito alla loro posizione.

L'apparente somiglianza tra i due concetti è dovuta solo al fatto che un sistema in cui una sola carica si può muovere è interamente specificato in un dato istante da tre sole coordinate, ovvero la posizione della carica stessa. L'energia potenziale appare in questo caso come un campo scalare perché è una funzione di $3$ variabili; se però si considera un sistema composto da un generico numero $N$ di cariche mobili, le variabili diventano $3N$ ($3$ per ogni carica) e la somiglianza scompare. Se tutte le cariche sono fisse tranne una, l'energia potenziale può essere influenzata solo dal moto della carica mobile, e la situazione si riconduce al moto di una carica in un campo elettrico, studiabile con il potenziale prodotto dal sistema di $N-1$ cariche fisse. Perciò, in quest'ultimo caso spesso la variazione di energia potenziale viene comunque indicata come $\Delta U (q)$.

Calcoliamo ora il valore dell'energia potenziale per sistemi discreti di cariche. Volendo costruire un sistema di due cariche, abbiamo che inizialmente non c'è interazione elettrica: il lavoro per portare la prima carica in un punto arbitrario è quindi 
\begin{equation}
L_{1} = 0
\end{equation}
Una volta posizionata questa carica, essa interagirà però con la seconda carica che vogliamo posizionare: il lavoro da compiere per portarla a distanza $r_{12}$ dalla prima sarà indipendente dal percorso scelto, essendo la forza elettrostatica conservativa, e pari a 
\begin{equation}
L_{2} = \int_{+\infty}^{r_{12}} \mathbf{F}_{ext} \cdot \mathbf{\dx{r}} = \int_{+\infty}^{r_{12}} \dfrac{1}{4 \pi \ezero} \dfrac{q_{1}q_{2}}{r_{12}^{2}} \rhat = \dfrac{1}{4 \pi \ezero} \dfrac{q_{1}q_{2}}{r_{12}} - U_{\infty} = U - U_{\infty}
\end{equation}
Dato che di solito si pone $U_{\infty} = 0$, abbiamo che $L = U$. Si noti che, costruendo il sistema al contrario, ovvero posizionando prima la seconda carica, si sarebbe ottenuto lo stesso risultato: $U_{12} = U_{21}$.
\begin{mdframed}
Con cariche ulteriori si usa il principio di sovrapposizione del potenziale, che fornisce la forma generale dell'energia potenziale di un sistema di $n$ cariche
\begin{equation} \label{eq:EnerTotSist}
U = \dfrac{1}{2} \sum_{i=1}^{n} q_{i} \left( \sum_{i \neq j} \dfrac{1}{4 \pi \ezero} \dfrac{q_{j}}{r_{ij}} \right)
\end{equation}
\end{mdframed}
dove la divisione per $2$ serve a compensare il fatto che la sommatoria conta ogni interazione due volte ($ij$ e $ji$), e inoltre si è tenuto conto del fatto che una carica (ferma) non interagisce con se stessa. 


Questa forma è generalizzabile a sistemi continui, per i quali
\begin{equation}
U = \dfrac{1}{2} \int_{\tau} \rho V \dx{\tau}
\end{equation}

\subsection{Altri campi elettrostatici}
\subsubsection{Anello uniformemente carico}
Si consideri un anello (una circonferenza) con centro nell'origine, di raggio $R$ e carica $Q$ uniformemente distribuita su di esso. La sua densità di carica lineare è 
\begin{equation}
\lambda = \dfrac{Q}{2 \pi R}
\end{equation}
Un tratto infinitesimo di circonferenza che sottende un angolo $\dx{\alpha}$ ha lunghezza $\dx{l} = R \dx{\alpha}$. Calcoliamo il campo prodotto da questa distribuzione sull'asse dell'anello, che consideriamo essere l'asse $z$. Per simmetria del sistema, le componenti ortogonali a $z$ dei campi infinitesimi generati da due tratti infinitesimi di anello diametralmente opposti si annullano a vicenda. Indicando con $\theta$ l'angolo fra la congiungente il punto in cui si calcola il campo e il tratto infinitesimo in esame, si ha quindi
\begin{align}
E = E_{z} 	&= \int \dx{E_{z}} \\
			&= \int \cos \theta \dx{E} \\
			&= \int \dfrac{1}{4 \pi \ezero} \dfrac{\dx{q}}{R^{2}+z^{2}} \cos \theta \\
			&= \dfrac{1}{4 \pi \ezero} \dfrac{1}{R^{2}+z^{2}} \int_{0}^{2 \pi} \dfrac{Q}{2 \pi R} \dfrac{z}{\sqrt{R^{2} + z^{2}}} R \dx{\alpha} \\
			&= \dfrac{1}{4 \pi \ezero} \dfrac{1}{R^{2}+z^{2}} \dfrac{z}{(R^{2} + z^{2})^{\frac{1}{2}}} \dfrac{Q}{2 \pi} \int_{0}^{2 \pi} \dx{\alpha} \\
			&= \dfrac{Qz}{4 \pi \ezero (z^{2}+ R^{2})^{\frac{3}{2}}}
\end{align}
\begin{mdframed}
Dunque sull'asse dell'anello, nel punto con coordinata $z$, si ha
\begin{equation}
\mathbf{E} = \dfrac{1}{4 \pi \ezero}\dfrac{Qz}{(z^{2}+ R^{2})^{\frac{3}{2}}} \khat
\end{equation}
\end{mdframed}

Per $z >> R$, questo valore tende a quello del campo generato da una carica puntiforme; per $z \rightarrow 0$ esso è lineare e si hanno dunque oscillazioni armoniche di una carica discorde attorno al centro dell'anello. Integrando tramite un cambiamento di variabile in $x := z^{2}+R^{2}$, si ottiene che il potenziale generato da questa distribuzione sull'asse $z$ coincide con quello generato da una carica $Q$ posta a distanza $\sqrt{z^{2}+ R^{2}}$:
\begin{mdframed}
\begin{equation}
V = \dfrac{1}{4 \pi \ezero} \dfrac{1}{\sqrt{z^{2}+ R^{2}}}.
\end{equation}
\end{mdframed}

\subsubsection{Disco uniformemente carico}
Si consideri un disco centrato nell'origine, di raggio $R$, con carica $Q$ uniformemente distribuita sulla sua superficie. La sua densità di carica superficiale è pertanto 
\begin{equation}
\sigma = \dfrac{Q}{\pi R^{2}}
\end{equation}
Il campo elettrico può essere calcolato per sovrapposizione di quelli generati dagli anelli infinitesimi che formano il disco stesso. Essi sono tutti paralleli e concordi (essendo tutti diretti secondo $\khat$). Si ha quindi
\begin{align}
E = \int \dx{E}	&= \int_{0}^{R} \dfrac{\sigma z}{2 \ezero} \dfrac{r \dx{r}}{(r^{2}+z^{2}=^{\frac{3}{2}}} \\
				&= \dfrac{\sigma z}{2 \ezero} \int_{0}^{R} \dfrac{r \dx{r}}{(r^{2}+z^{2}=^{\frac{3}{2}}} \\
				&= \dfrac{\sigma z}{2 \ezero} \left[ - \dfrac{1}{\sqrt{r^2+z^2}} \right]_{0}^{R} \\
				&= \dfrac{\sigma }{2 \ezero} \left[ \mathrm{sgn} z - \dfrac{z}{\sqrt{R^2+z^2}} \right]
\end{align}
Dunque sull'asse del disco, nel punto con coordinata $z$, si ha
\begin{mdframed}
\begin{equation}
\mathbf{E} = \dfrac{\sigma}{2 \ezero} \left[ \mathrm{sgn} z - \dfrac{z}{\sqrt{R^2+z^2}} \right] \khat
\end{equation}
\end{mdframed}
Si noti che per $R \rightarrow \infty$ il campo tende a quello generato da un piano uniformemente carico.

\subsection{Dipoli elettrici}
\subsubsection{Potenziale e campo elettrostatico in ogni punto dello spazio}
Prendiamo un dipolo elettrico composto da due cariche $\pm Q$ separate da una distanza (fissa) $A$, disposto lungo l'asse $z$ e con centro nell'origine. Consideriamo un punto $P$ con vettore posizione $\mathbf{r} = (x, y, z)$, che forma un angolo $\theta$ con l'asse $z$, e i vettori $\mathbf{r}_{+}$ e $\mathbf{r}_{-}$ che vanno a $P$ rispettivamente dalla carica positiva e da quella negativa del dipolo. Supponiamo inoltre che valga $r \gg A$, che per la maggio parte dei dipoli è un'approssimazione ragionevole. Il potenziale di $P$ sarà la somma di quelli generati dalle due cariche:
\begin{equation}
V(P) = \dfrac{Q}{4 \pi \ezero} \left( \dfrac{1}{r_{+}} - \dfrac{1}{r_{-}} \right) = \dfrac{Q}{4 \pi \ezero} \left( \dfrac{r_{-} - r_{+}}{r_{+}r_{-}} \right)
\end{equation}
Poiché $r \gg A$, $\mathbf{r}_{+} = \mathbf{r} - \frac{A}{2} \ihat$ e $\mathbf{r}_{-} = \mathbf{r} + \frac{A}{2} \ihat$, abbiamo $r_{+} \approx r_{-} \approx r$ e inoltre $r_{-} - r_{+} \approx A \cos \theta$. Sostituendo, abbiamo che
\begin{equation}
V(P) = \dfrac{1}{4 \pi \ezero} Q \dfrac{A \cos \theta}{r^{2}} = \dfrac{1}{4 \pi \ezero} \dfrac{|\mathbf{P}| \cos \theta}{r^{2}}  = \dfrac{1}{4 \pi \ezero} \dfrac{|\mathbf{P}| \cos \theta}{r^{2}} \dfrac{r}{r} = \dfrac{1}{4 \pi \ezero} \dfrac{\mathbf{P} \cdot \mathbf{r}}{r^{3}}
\end{equation}
dove $\mathbf{P} = \mathbf{A} Q$ è il momento di dipolo. Si noti che, come per il campo elettrico, il potenziale di un dipolo non dipende direttamente dalle cariche ma dal momento di dipolo, e decresce in modulo più rapidamente di quello prodotto da una carica puntiforme, per via della compensazione fra il potenziale della carica positiva e di quella negativa.

Ricaviamo ora il campo elettrico componente per componente, derivando il potenziale. Abbiamo che 
\begin{align}
E_{\parallel} = E_{z} = - \parder{V}{z}	&= \parder{}{z} \left( \dfrac{1}{4 \pi \ezero} \dfrac{|\mathbf{P}|\khat \cdot (x \ihat + y \jhat + z \khat)}{(x^2 + y^2 + z^2)^{\frac{3}{2}}} \right)\\
										&= \dfrac{1}{4 \pi \ezero} \parder{}{z} \left( \dfrac{Pz}{r^3} \right)  \\
										&= - \dfrac{|\mathbf{P}|}{4 \pi \ezero} \left[ \dfrac{1}{r^3} - \dfrac{3}{2} \dfrac{z \cdot 2z}{r^5} \right] \\
										&= \dfrac{|\mathbf{P}|}{4 \pi \ezero} \left[ \dfrac{3 r^2 \cos^{2} \theta}{r^5} - \dfrac{1}{r^3} \right] \\
										&= \dfrac{1}{4 \pi \ezero} \dfrac{|\mathbf{P}|}{r^3}(3 \cos^2 \theta - 1)
\end{align}
Analogamente, si mostra che 
\begin{align}
E_{x} &= \dfrac{1}{4 \pi \ezero} \dfrac{|\mathbf{P}|}{r^3} 3zx \\
E_{y} &= \dfrac{1}{4 \pi \ezero} \dfrac{|\mathbf{P}|}{r^3} 3zy
\end{align}
e dunque 
\begin{equation}
E_{\perp} = \sqrt{E_{x}^{2} + E_{y}^{2}} = \dfrac{1}{4 \pi \ezero} \dfrac{|\mathbf{P}|}{r^3}\cdot 3 \cos \theta \sin \theta
\end{equation}
Nei casi particolari in cui $\theta = 0$ o $\theta = \frac{\pi}{2}$, ovvero quando $P$ si trova sulla retta del dipolo o sul suo asse, la componente normale al dipolo è nulla; quella parallela vale invece rispettivamente $\frac{1}{2 \pi \ezero} \frac{|\mathbf{P}|}{r^3}$ e $- \frac{1}{4 \pi \ezero} \frac{|\mathbf{P}|}{r^3}$. Nel secondo caso, il risultato coincide con quello ricavato nella sezione \ref{subsubsec:DipAsse}.

\subsubsection{Moto di un dipolo in un campo elettrostatico uniforme}
Consideriamo ora un dipolo costituito da due cariche $\pm q$, con raggio vettore $\mathbf{a}$ dalla negativa alla positiva, immerso in un campo elettrostatico uniforme $\mathbf{E}$. Sia $\mathbf{r}$ il vettore posizione della carica negativa e supponiamo ancora una volta che $r \gg a$. La forza totale a cui è sottoposto il dipolo è
\begin{equation}
\sum \mathbf{F} = \mathbf{F}_{+} + \mathbf{F}_{-} = \mathbf{E} (q - q) = 0
\end{equation}
Dunque il centro di massa del dipolo non accelera.
Per quanto riguarda invece il momento torcente che agisce sul sistema, calcoliamolo rispetto all'origine: siano $\mathbf{r}_{+}$ ed $\mathbf{r}_{-}$ i vettori posizione delle cariche. Abbiamo che
\begin{equation}
\sum \mathbf{M}_{O} = \mathbf{r}_{+} \wedge \mathbf{F}_{+} + \mathbf{r}_{-} \wedge \mathbf{F}_{-} = \mathbf{r}_{+} \wedge q \mathbf{E} - \mathbf{r}_{-} \wedge q \mathbf{E} = (\mathbf{r}_{+} - \mathbf{r}_{-}) \wedge q \mathbf{E} = \mathbf{a} \wedge q \mathbf{E} = \mathbf{P} \wedge \mathbf{E}
\end{equation}
Quindi $\sum \mathbf{M}_{O} = 0$ se e solo se l'angolo $\theta$ fra $\mathbf{a}$ ed $\mathbf{r}$ è uguale a $0^{\circ}$ o a $180^{\circ}$, vale a dire se il dipolo è disposto parallelamente al campo. D'altra parte, allo stesso risultato si poteva giungere considerando l'energia potenziale del sistema. Si ha infatti che essa è uguale al lavoro compiuto da una forza esterna per assemblare il sistema, vale a dire per mettere la carica negativa in posizione $\mathbf{r}$ e successivamente quella positiva in $\mathbf{r} + \mathbf{a}$:
\begin{align}
U = L_{ext} 	&= \int_{+\infty}^{\mathbf{r+a}} \mathbf{F}_{ext} \cdot \mathbf{\dx{s}} + \int_{+\infty}^{\mathbf{r}} \mathbf{F}_{ext} \cdot \mathbf{\dx{s}} \\
				&= q V(\mathbf{r + a}) - q V(+\infty) - qV(\mathbf{r}) + qV(+\infty) + U_{dip} \\
				&= q V(\mathbf{r + a}) - qV(\mathbf{r}) + U_{dip} \\
				&= q[V(x+a_{x}, y+a_{y}, z+a_{z}) - V(x,y,z)] + U_{dip}\\
				&= q \dx{V} + U_{dip}\\
				&= q (\bnabla V \cdot \mathbf{a}) + U_{dip}\\
				&= -q (\mathbf{a} \cdot \mathbf{E}) + U_{dip} \\
				&= - \mathbf{P} \cdot \mathbf{E} + U_{dip} \\
				&= - P E \cos \theta + U_{dip}
\end{align}
dove $U_{dip}$ è una costante corrispondente al lavoro svolto dal campo della carica negativa su quella positiva durante l'avvicinamento delle due. Il sistema è all'equilibrio quando $U$ è minima, ovvero quando $\cos \theta = 1$ ($U$ è negativa!), il che avviene per $\theta = 0^{\circ}$ o $\theta = 180^{\circ}$, il che conferma il risultato ottenuto sopra. Si noti che il risultato $U = - \mathbf{P} \cdot \mathbf{E} + U_{dip}$ è di validità generale, non solo per campi uniformi; nel caso generale, però, $U$ potrà dipendere anche dalla posizione del dipolo, che dunque traslerà.

\subsubsection{Sistemi di cariche}
Consideriamo un sistema di cariche $q_{1},...,q_{n}$ con posizioni $\mathbf{d}_{1},...,\mathbf{d}_{n}$. Sia $d$ la lunghezza caratteristica del sistema. Calcoliamo il potenziale generato da questo sistema nel punto $P$, con vettore posizione $\mathbf{r}$, nell'assunzione che $r \gg d$. Chiamiamo $\mathbf{r}_{i} = \mathbf{r} - \mathbf{d}_{i}$. Usiamo il principio di sovrapposizione e il fatto che, poiché $r \gg d \sim d_{i}$, $r_i \approx r - d_{i} \cos \theta$ e $d_{i}^2 \approx 0$: 
\begin{align}
V(P) 	&= \dfrac{1}{4 \pi \ezero} \sum \dfrac{q_{i}}{r_{i}} \\
		&= \dfrac{1}{4 \pi \ezero} \sum \dfrac{q_{i}}{r - d_{i} \cos \theta_{i}} \\
		&= \dfrac{1}{4 \pi \ezero} \sum \dfrac{q_{i}}{r - \mathbf{d}_{i} \cdot \rhat} \\
		&= \dfrac{1}{4 \pi \ezero} \sum \dfrac{q_{i}}{r - \mathbf{d}_{i} \cdot \rhat} \dfrac{r + \mathbf{d}_{i} \cdot \rhat}{r + \mathbf{d}_{i} \cdot \rhat}\\
		&= \dfrac{1}{4 \pi \ezero} \sum q_{i} \dfrac{r + \mathbf{d}_{i} \cdot \rhat}{r^2 - (\mathbf{d}_{i} \cdot \rhat)^2}\\
		&\approx \dfrac{1}{4 \pi \ezero} \sum \dfrac{q_{i}}{r^2} (r + \mathbf{d}_{i} \cdot \rhat) \\
		&= \dfrac{1}{4 \pi \ezero} \sum q_{i} \dfrac{r}{r^2} + \dfrac{1}{4 \pi \ezero} \sum q_{i} \dfrac{\mathbf{d}_{i} \cdot \rhat}{r^2} \\
		&= \dfrac{1}{4 \pi \ezero} \dfrac{Q}{r} + \dfrac{1}{4 \pi \ezero}  \dfrac{(\sum q_{i}\mathbf{d}_{i}) \rhat}{r^2} \\
\end{align}
dove $Q$ è la carica totale del sistema. Si definisce il vettore \emph{momento di dipolo} (per un sistema di $N$ cariche) come
\begin{mdframed}
\begin{equation}
\mathbf{P} = \sum_{i} q_{i} \mathbf{d}_{i}
\end{equation}
\end{mdframed}
e si esprime quindi il potenziale generato da un sistema di cariche come una somma di due termini: il \emph{termine di monopolo} $V_0$, pari al potenziale che produrrebbe la carica totale del sistema se fosse posta interamente nell'origine, e il \emph{termine di dipolo} $V_{dip}$, dipendente invece dalla distribuzione di carica, che rende possibile al sistema produrre una differenza di potenziale anche se $Q = 0$:
\begin{mdframed}
\begin{equation}
V(P) = \dfrac{1}{4 \pi \ezero} \dfrac{Q}{R} + \dfrac{1}{4 \pi \ezero} \dfrac{\mathbf{P} \cdot \rhat}{r^2} := V_{0} + V_{dip}
\end{equation}
\end{mdframed}
Facciamo una stima del valore relativo di questi due termini: stimando $|\mathbf{P} \cdot \rhat| \sim |\mathbf{P}|$ $= |\sum q_i \mathbf{d}_i| \sim |\sum q_{i} d| = |Qd|$, abbiamo che
\begin{equation}
\dfrac{|V_{dip}|}{|V_0|} = \dfrac{|\mathbf{P} \cdot \rhat|}{r^2} \dfrac{r}{|Q|} \sim \dfrac{Qd}{Qr} = \dfrac{d}{r} \ll 1
\end{equation}
per ipotesi. Dunque $V_{dip}$ è apprezzabile solo per sistemi neutri (per i quali non vale più questa stima).

$\mathbf{P}$ ha inoltre la particolare caratteristica di dipendere dal sistema di riferimento solo nel caso in cui il sistema non sia neutro: se $Q = 0$, infatti, cambiando l'origine da $O$ a $O'$ (spostandola di un vettore $\mathbf{R})$, si ha
\begin{equation}
\mathbf{P}_{O} = \sum q_{i} \mathbf{d}_{i} = \sum q_{i} (\mathbf{R} + d'_{i}) = Q\mathbf{R} + \mathbf{P'} = \mathbf{P'}
\end{equation}

Vediamo ora com'è possibile ricondurre un qualsiasi sistema neutro di cariche a un dipolo: si definiscano i vettori posizione medi pesati delle cariche positive e negative
\begin{align}
\mathbf{d}^{+} &:= \dfrac{\sum |q_{i}^{+}|\textbf{d}_{i}^{+}}{\sum |q_{i}^{+}|} \\
\mathbf{d}^{-} &:= \dfrac{\sum |q_{i}^{-}|\textbf{d}_{i}^{-}}{\sum |q_{i}^{-}|}
\end{align}
Si ha quindi che $\mathbf{P} = |Q^{+}| \mathbf{d}^{+} - |Q^{-}| \mathbf{d}^{-}$. Se il sistema è neutro, $|Q^{+}| = |Q^{-}| := Q$ e dunque 
\begin{equation}
\mathbf{P} = Q (\mathbf{d^{+}} - \mathbf{d^{-}}) = 0 \ \Leftrightarrow \ \mathbf{d^{+}} = \mathbf{d^{-}}
\end{equation}
\begin{mdframed}
Definendo il vettore $\boldsymbol{\mathbf{\delta}} := \mathbf{d^{+}} - \mathbf{d^{-}}$, abbiamo che
\begin{equation}
\mathbf{P} = 0 \ \Leftrightarrow \ \boldsymbol{\mathbf{\delta}} = 0
\end{equation}
\end{mdframed}
Ogni distribuzione neutra può essere ricondotta a un dipolo costituito da cariche $\pm Q$ separate dal vettore $\boldsymbol{\mathbf{\delta}}$. Se $\boldsymbol{\mathbf{\delta}} = 0$, le due cariche coincidono e i loro potenziali si annullano a vicenda. È quello che succede nell'atomo di idrogeno, ma non nella molecola d'acqua, che infatti è \emph{polare}. In realtà, se anche il temine di dipolo è nullo, non sempre il potenziale generato dal dipolo è costante: esso può essere approssimato con termini di ordine superiore (quadrupolo, ..., $2^{n}$-polo), ognuno trascurabile rispetto al precedente.
\end{document}
