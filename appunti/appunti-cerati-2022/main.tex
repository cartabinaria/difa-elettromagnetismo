\documentclass{article}
 
\usepackage[T1]{fontenc}
\usepackage[italian]{babel}
\usepackage{geometry}
 
\usepackage{amsmath}
\usepackage{amsfonts}
\usepackage{amssymb}
\usepackage{siunitx}

\usepackage{amsthm}
\usepackage{mdframed}

\usepackage{textgreek}

\usepackage{tikz}

\usepackage[shortlabels]{enumitem}

\usepackage{hyperref}
\usepackage{subfiles}
\usepackage{xr}

\theoremstyle{plain}
\newmdtheoremenv{thm}{Teorema}[section]

\theoremstyle{plain}
\newtheorem{fact}[thm]{Osservazione}

\theoremstyle{definition}
\newtheorem{dfn}[thm]{Def}

\numberwithin{equation}{section}

\renewcommand\qedsymbol{QED}

\newcommand{\dx}[1]{\mathrm{d} #1}
\newcommand{\ezero}{\varepsilon _0}

\newcommand{\ihat}{\boldsymbol{\hat{\textbf{\i}}}}
\newcommand{\jhat}{\boldsymbol{\hat{\textbf{\j}}}}
\newcommand{\khat}{\boldsymbol{\hat{\textbf{k}}}}
\newcommand{\rhat}{\boldsymbol{\hat{\textbf{r}}}}
\newcommand{\versor}[1]{\boldsymbol{\hat{\mathbf{#1}}}}
\newcommand{\bnabla}{\boldsymbol{\nabla}}
\newcommand{\parder}[3][]{\frac{\partial^{#1} #2}{\partial #3 ^{#1}}}

\geometry{margin=1.1in}

\externaldocument[BT-]{build/bilancia-torsione}
\externaldocument[N-]{build/nabla}
\externaldocument[ES-]{build/elettrostatica}
\externaldocument[G-]{build/gauss}

\begin{document}
\begin{titlepage} \label{title}

\title{Elettromagnetismo -- UniBo}
\author{Alessandro Cerati}
\date{A.A. 2022/2023}
\maketitle

\begin{center}
\begin{small}
Basato sulle lezioni dei prof. Zoccoli, Rinaldi e Spighi \\
\end{small}
\vspace{100pt}
\begin{huge}
\textsc{\textbf{EARLY ACCESS SOLO ELETTRICIT\'{A}} \\} 
\end{huge}
L'Autore declina ogni responsabilità per eventuali stronzate scritte qui dentro.\\
\end{center}

\end{titlepage}

\tableofcontents
\newpage

\subfile{build/intro}

\subfile{build/elettrostatica}

\subfile{build/gauss}

\subfile{build/conduttori}

\appendix

\subfile{build/bilancia-torsione}

\subfile{build/nabla}

\end{document}