\documentclass[./main.tex]{subfiles}
\begin{document}
\section{Introduzione}
\subsection{Notazione usata}
Per le definizioni delle grandezze ho usato il simbolo $:=$, lasciando $\equiv$ per l'uguaglianza identica (per esempio di due funzioni su tutto il loro dominio). I vettori sono indicati con il grassetto ($\mathbf{F}$) e le loro componenti con il carattere normale con apice o pedice ($F_{x}$ o $F^{\perp}$). I loro moduli con il simbolo in carattere normale ($F$) o, dove ci sarebbe un conflitto di notazione, con il vettore fra sbarre ($|\mathbf{F}|$); in ogni caso, essi sono sempre non negativi. Gli integrali di superficie e di percorso sono indicati con il dominio di integrazione in basso a destra.

\textsc{\textbf{Questa è una versione parziale e non rivista: mancano le figure e ci sono integrazioni dal libro solo nella prima parte della sezione sull'elettrostatica. Il resto è solo preso dai miei appunti delle lezioni, e non è stato troppo ricontrollato.}}
\subsection{Introduzione all'elettromagnetismo}
Come meccanica e termodinamica classica, l'elettromagnetismo si occupa di effetti macroscopici. A differenza di queste, tuttavia, questi effetti sono spiegabili \emph{solo} in termini microscopici. Inoltre, esso è la prima introduzione vera e propria al concetto di \emph{campo} in quanto entità indipendente dalle interazioni, che sarà poi fondamentale in meccanica quantistica.
\end{document}