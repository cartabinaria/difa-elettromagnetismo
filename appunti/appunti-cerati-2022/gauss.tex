\documentclass[main.tex]{subfiles}
\begin{document}

\section{Legge di Gauss}
\subsection{Definizioni preliminari}
\paragraph{Angolo solido}
Matematicamente, un angolo è la porzione di piano compresa tra due semirette, e si misura (in radianti) tramite il rapporto tra la lunghezza di un qualsiasi arco di circonferenza che esso sottende e il raggio della corrispondente circonferenza. Analogamente, si definisce \emph{angolo solido} la porzione di spazio compresa nel cono infinito individuato da 3 rette. Esso si misura in \emph{steradianti}, ovvero rapportando l'area $\Sigma$ di una calotta sferica da esso sottesa (ma non per forza con centro nel vertice) al raggio $r$ della stessa sfera, elevato al quadrato.
\begin{equation}
\Omega := \dfrac{\Sigma}{r^{2}}
\end{equation}
Dato che l'area di una sfera vale $4 \pi r^{2}$, si ha che $\Omega \in [0, 4 \pi]$.
\paragraph{Flusso}
Si consideri una superficie arbitraria $\Sigma$. Si consideri una sua porzione infinitesima di superficie $\dx{S}$ e sia $\versor{u}$ il versore normale ad essa e orientato convenzionalmente (verso l'esterno se $\Sigma$ è chiusa, secondo la regola della mano destra se è aperta). Si definisce \emph{vettore superficie} il vettore
\begin{equation}
\mathbf{dS} := \dx{S}\ \versor{u}
\end{equation}

\begin{mdframed}
Si consideri ora un campo vettoriale $\mathbf{E}$ che formi in $\dx{S}$ un angolo $\theta$ con $\versor{u}$. Per analogia al flusso di un fluido attraverso una superficie, si definisce \emph{flusso (infinitesimo)} di $\mathbf{E}$ attraverso $\dx{S}$ la grandezza scalare
\begin{equation}
\dx{\Phi}(\mathbf{E}) := \mathbf{E} \cdot \mathbf{dS}
\end{equation}
Si noti che questa può essere positiva, negativa o nulla. Il flusso attraverso una superficie finita $\Sigma$ si ottiene integrando su di essa.
\begin{equation}
\Phi(\mathbf{E}) = \int_{\Sigma} \dx{\Phi}(\mathbf{E}) = \int_{\Sigma} \mathbf{E} \cdot \dx{S}
\end{equation}
\end{mdframed}

\subsection{Forma integrale}
\begin{thm}[Legge di Gauss -- forma integrale]
Si consideri una superficie chiusa $\Sigma$ e sia $Q_{T}$ la carica totale racchiusa al suo interno. Allora si ha che 
\begin{equation}
\oint_{\Sigma} \mathbf{E} \cdot \mathbf{dS} = \dfrac{Q_{T}}{\ezero}
\end{equation}
dove $\mathbf{E}$ è il campo elettrostatico e i $\mathbf{dS}$ sono i vettori superficie infinitesimi in ogni punto di $\Sigma$, e $\ezero$ è la costante dielettrica del vuoto.
\end{thm}
\begin{proof}
I contributi al flusso attraverso $\dx{S}$ possono essere forniti da cariche interne o esterne. 

Considerando una carica esterna $Q$, si ponga l'origine del sistema di riferimento coincidente con essa. Si scelga un angolo solido infinitesimo $\dx{\Omega}$. Se esso non interseca $\Sigma$ o vi è tangente, è immediato notare che il suo contributo al flusso è nullo. In caso contrario, esso interseca $\Sigma$ in due superfici infinitesime distinte $\dx{S}_{1}$ e $\dx{S}_{2}$, nelle posizioni $\mathbf{r}_{1}$ ed $\mathbf{r}_{2}$, sulle quali il campo vale rispettivamente $\mathbf{E}_{1}$ ed $\mathbf{E}_{2}$ e forma con i vettori superficie angoli $\theta_{1}$ e $\theta_{2}$. Si ha che
\begin{align}
\dx{\Phi}_{2} 	&= \mathbf{E}_{2} \cdot \mathbf{dS}_{2} = E_{2} \dx{S}_{2} \cos \theta_{2} = E_{2} r_{2}^{2} \dfrac{\dx{S}_{2} \cos \theta_{2}}{r_{2}^{2}} = E_{2} r_{2}^{2} \dx{\Omega} \\
				&= \dfrac{1}{4 \pi \ezero} Q \dx{\Omega} 
\end{align}
e analogamente si può ottenere che 
\begin{equation}
\dx{\Phi}_{1} = - \dfrac{1}{4 \pi \ezero} Q \dx{\Omega} 
\end{equation}
e dunque
\begin{equation}
\dx{\Phi}_{\mathrm{ext}} = \dx{\Phi}_{1} + \dx{\Phi}_{2} = 0
\end{equation}
Il ragionamento si può ripetere su tutto l'angolo solido, dunque l'apporto totale delle cariche esterne è nullo.

Consideriamo ora invece una carica $Q$ interna a $\Sigma$ e poniamo in essa l'origine degli assi. Si consideri ancora una volta un angolo solido infinitesimo $\dx{\Omega}$. Questa volta, esso interseca sicuramente $\Sigma$ in un'unica superficie infinitesima $\dx{S}$. Ragionando come sopra, si ottiene sempre
\begin{equation}
\dx{\Phi} = - \dfrac{1}{4 \pi \ezero} Q \dx{\Omega} 
\end{equation}
Integrando su tutto l'angolo solido si ottiene che il flusso dovuto a $Q$ è
\begin{equation}
\Phi = \dfrac{Q}{4 \pi \ezero} \int_{0}^{4 \pi} \dx{\Omega} = \dfrac{Q}{\ezero}
\end{equation}
Per la proprietà distributiva, si ha quindi che il flusso dovuto alla totalità delle cariche interne a $\Sigma$ è 
\begin{equation}
\Phi_{\mathrm{int}}= \dfrac{Q_{T}}{\ezero}
\end{equation}
Il flusso totale attraverso $\Sigma$ è quindi 
\begin{equation}
\Phi(\mathbf{E}) = \Phi_{\mathrm{ext}} + \Phi_{\mathrm{int}} = 0 + \dfrac{Q_{T}}{\ezero} = \dfrac{Q_{T}}{\ezero}
\end{equation}
\end{proof}

Se la carica è distribuita in maniera continua, si ha invece 
\begin{equation}
Q_{T} = \int \dx{q} 
\end{equation}
dove $\dx{q}$ è calcolata usando la distribuzione appropriata (volumetrica, superficiale o lineare).

Dalla legge di Gauss segue che il flusso del campo prodotto da un dipolo attraverso una superficie chiusa che contiene interamente quest'ultimo è nullo.

La dimostrazione si può ripetere per qualsiasi campo radiale con modulo proporzionale a $r^-2$; in particolare, può essere applicata al campo gravitazionale. Si noti che questa proprietà è esclusiva dei campi il cui modulo è proporzionale specificamente a $r^-2$: conferme sperimentali della sua validità costituiscono quindi conferme sperimentali della legge di Coulomb.

Dalla legge di Gauss scritta in forma integrale si possono ricavare solo informazioni "mediate" sulla superficie $\Sigma$. $\mathbf{E}$ si può ricavare direttamente solo nel caso in cui da considerazioni di simmetria si riesca a trovare una superficie $\Sigma$ sulla quale esso ha modulo costante e direzione $\versor{E}$ nota (vedi la sezione \ref{subsec:GaussE}). In tal caso, infatti, si ha
\begin{align}
\dfrac{Q_{T}}{\ezero} 	&= \oint_{\Sigma} \mathbf{E} \cdot \mathbf{\dx{S}} \\
						&= \oint_{\Sigma} E \cos \theta \ \dx{S} \\
						&= E \oint_{\Sigma} \cos \theta \ \dx{S} \\
						&\Rightarrow E = \dfrac{Q_{T}}{\ezero \oint_{\Sigma} \cos \theta \dx{S}} \\
						&\Rightarrow \mathbf{E} = \dfrac{Q_{T}}{\ezero \oint_{\Sigma} \cos \theta \dx{S}} \versor{E}
\end{align}
dove $\theta$ è l'angolo tra $\mathbf{\dx{S}}$ e $\versor{E}$; il calcolo risulta molto più semplice nei (limitati) casi in cui anche $\cos \theta$ è costante su $\Sigma$. 
Un altro modo per ottenere informazioni puntuali sul campo elettrostatico è la \emph{forma differenziale} della legge di Gauss.

\subsection{Forma differenziale}
\begin{thm}[Legge di Gauss -- forma differenziale]
In ogni punto dello spazio vale 
\begin{equation}
\bnabla \cdot \mathbf{E} = \dfrac{\rho}{\ezero}
\end{equation}
dove $\mathbf{E}$ e $\rho$ sono rispettivamente il valore del campo elettrostatico e la densità di carica in quel punto, e $\ezero$ è la costante dielettrica del vuoto.
\end{thm}
\begin{proof} 
Sia $\Sigma$ un'arbitraria superficie chiusa e $V_{\Sigma}$ il volume da essa racchiuso, la legge di Gauss per distribuzioni continue fornisce
\begin{equation}
\oint_{\Sigma} \mathbf{E} \cdot \mathbf{\dx{S}} = \dfrac{1}{\ezero} \int_{V_{\Sigma}} \rho \dx{V}
\end{equation}
D'altra parte, il teorema della divergenza, valido per qualsiasi campo vettoriale, afferma che 
\begin{equation}
\oint_{\Sigma} \mathbf{E} \cdot \mathbf{\dx{S}} = \int_{V_{\Sigma}} (\bnabla \cdot \mathbf{E}) \dx{V}
\end{equation}
e dunque si ha (portando $\ezero$ dentro l'integrale)
\begin{equation}
\int_{V_{\Sigma}} \dfrac{\rho}{\ezero} \dx{V} = \int_{V_{\Sigma}} (\bnabla \cdot \mathbf{E}) \dx{V}
\end{equation}
Dato però che $\Sigma$ è completamente arbitraria, ciò implica che gli integrandi siano identicamente uguali:
\begin{equation}
\dfrac{\rho}{\ezero} = \bnabla \cdot \mathbf{E} 
\end{equation}
\end{proof}

\subsection{Attraversamento di superfici cariche}
Sia corpi conduttori che isolanti spesso presentano, in prossimità delle loro superfici, una densità di carica di spessore trascurabile, che può essere considerata a tutti gli effetti superficiale. Sui due lati di una distribuzione di questo tipo, la componente normale e quella tangenziale del campo elettrostatico presentano proprietà che possono essere dedotte dalla legge di Gauss.
\paragraph{Componente normale} \label{par:CompNormSupCar}
Consideriamo una superficie carica con densità in generale non uniforme $\sigma$ e applichiamo il teorema di Gauss ad una superficie cilindrica infinitesima $\Sigma$ di base $\dx{S}$ e altezza $\dx{h}$, con asse ortogonale alla superficie in quel punto. Essendo infinitesimo, il valore di $\mathbf{E}$ sulle basi superiore e inferiore, rispettivamente $\mathbf{E}_{U}$, $\mathbf{E}_{D}$ ed $\mathbf{E}_{L}$ è univoco, così come il valore di $\sigma$ all'interno del cilindro. I flussi infinitesimi attraverso le basi sono quindi ben definiti; tuttavia, non lo è il flusso infinitesimo attraverso la superficie laterale\footnote{Il campo elettrico non è per forza ortogonale a \emph{questa} superficie: quella legge vale solo per i conduttori.}, a causa del fatto che il versore $\mathbf{\dx{S}}_{L}$ non è univoco. Il problema può però essere risolto mandando $\dx{h}$ a $0$ più rapidamente: così facendo, il contributo della superficie laterale al flusso diventa trascurabile. La carica contenuta in $\Sigma$ è $\frac{\sigma \dx{S}}{\ezero}$; la legge di Gauss fornisce 
\begin{equation}
\dfrac{\sigma \dx{S}}{\ezero} = \mathbf{E}_{U} \cdot \mathbf{\dx{S}}_{U} + \mathbf{E}_{D} \cdot \mathbf{\dx{S}}_{D} + \mathbf{E}_{L} \cdot \mathbf{\dx{S}}_{L} \longrightarrow \mathbf{E}_{U} \cdot \mathbf{\dx{S}}_{U} + \mathbf{E}_{D} \cdot \mathbf{\dx{S}}_{D} = (E_{U}^{\perp} - E_{D}^{\perp}) \dx{S}
\end{equation}
Infatti, i prodotti scalari degli $\mathbf{E}$ con i $\dx{S}$ altro non sono che le loro proiezioni sulla normale alla superficie, avendo cura di cambiare di segno una delle due dato che i versori normali alle basi sono antiparalleli. Si noti che le $E^{\perp}$ sono \emph{componenti}, e quindi hanno un segno.
\begin{mdframed}
Attraversando \emph{dall'interno all'esterno} una superficie carica in un punto con densità di carica (locale) $\sigma$ si osserva quindi una discontinuità nella componente del campo elettrostatico ortogonale alla superficie pari a
\begin{equation}
\Delta E_{\perp} = \dfrac{\sigma}{\ezero}
\end{equation}
dove $\ezero$ è la costante dielettrica del vuoto.
\end{mdframed}

\paragraph{Componente parallela}
Si consideri la stessa superficie del paragrafo precedente, ma questa volta si prenda un rettangolo infinitesimo $\Gamma$ con un lato $\dx{l}$ parallelo alla superficie e un lato $\dx{h}$ ad essa normale. Si mandi $\dx{h}$ a $0$. Il campo elettrostatico è conservativo, quindi la sua circuitazione lungo $\Gamma$ deve essere nulla. Si ha quindi 
\begin{equation}
0 = \mathbf{E}_{1} \cdot \mathbf{\dx{l}}_{1} + \mathbf{E}_{2} \cdot \mathbf{\dx{h}}_{2} +\mathbf{E}_{3} \cdot \mathbf{\dx{l}}_{3} +\mathbf{E}_{4} \cdot \mathbf{\dx{h}}_{4} \longrightarrow \mathbf{E}_{1} \cdot \mathbf{\dx{l}}_{1} + \mathbf{E}_{3} \cdot \mathbf{\dx{l}}_{3} = (E_{1}^{\parallel} - E_{3}^{\parallel})
\end{equation}
con ragionamento analogo a sopra.
\begin{mdframed}
Dunque, attraversando una superficie carica, la componente tangente del campo \emph{elettrostatico} non cambia.
\begin{equation}
\Delta E_{\parallel} = 0
\end{equation}
\end{mdframed}
Per inciso, il fatto che solo una delle componenti del campo sia discontinua attraversando una superficie carica è la causa del fenomeno della rifrazione.

\subsection{Calcolo di \texorpdfstring{$\mathbf{E}$}{E} tramite la legge di Gauss} \label{subsec:GaussE}
\subsubsection{Distribuzione lineare uniforme infinita}
Torniamo alla distribuzione lineare uniforme infinita e calcoliamo il campo elettrostatico con il nuovo metodo. Si scelga il sistema di riferimento in modo che il filo giaccia sull'asse $z$. Il sistema ha chiaramente simmetria cilindrica, dunque $E$ deve essere funzione solo della distanza $r$ dal filo e inoltre deve essere $\versor{E} = \rhat$. Si consideri una superficie cilindrica $\Sigma$ di altezza $h$ e raggio $r$ con asse coincidente col filo. Il flusso del campo elettrostatico attraverso il cilindro è la somma dei flussi attraverso le superfici di base $S_{U}$ e $S_{D}$ e di quello attraverso la superficie laterale $S_{L}$. Per costruzione, si ha sempre $\mathbf{\dx{S}}_{U} \parallel \mathbf{\dx{S}}_{D} \parallel \versor{k} \perp \rhat \parallel \versor{E}$, dunque il flusso attraverso le basi è nullo. Il flusso del campo coincide quindi con quello attraverso $S_{L}$; in particolare, dato che si ha sempre $\mathbf{\dx{S}}_{L} \parallel \rhat \parallel \versor{E}$ e come detto sopra $E$ è costante su $S_{L}$,
\begin{equation}
\oint_{S_{L}} \mathbf{E} \cdot \mathbf{\dx{S}}_{L} = E \cos 0 \oint_{S_{L}} \dx{S_{L}} = E \cdot 2 \pi h
\end{equation}
La carica totale all'interno di $\Sigma$, avendo il filo densità di carica uniforme $\lambda$, è
\begin{equation}
Q_{T} = \lambda h
\end{equation}
La legge di Gauss afferma dunque che
\begin{equation}
\dfrac{\lambda}{h} = E \cdot 2 \pi h
\end{equation} 
dalla quale, riordinando i membri e moltiplicando per $\versor{E}$, si ricava che
\begin{equation}
\mathbf{E}(P) = \dfrac{1}{2 \pi \ezero} \dfrac{\lambda}{r} \rhat 
\end{equation}
Abbiamo ritrovato il risultato della sezione \ref{ES-subsubsec:DistrLinUnifSovrapp}.

\subsubsection{Distribuzione planare uniforme infinita}
Calcoliamo il campo elettrico prodotto da un piano infinito uniformemente carico. Si scelga il sistema di riferimento in modo che l'asse $x$ sia ortogonale al piano e l'origine giaccia su quest'ultimo. Il sistema è simmetrico per riflessione intorno al piano stesso, dunque se $\versor{n}$ è il versore normale al piano si ha $\versor{E} \parallel \versor{n}$, e inoltre $\mathbf{E} (x) = - \mathbf{E} (-x)$. Si consideri una superficie $\Sigma$ cilindrica di raggio $r$ e altezza qualsiasi, con asse l'asse $x$ e divisa a metà dal piano. Con considerazioni analoghe a prima, si nota che il flusso attraverso la superficie laterale è nullo, mentre $\mathbf{E}$ è costante e ortogonale al vettore superficie sulle due basi $S_{+}$ e $S_{-}$. Dunque
\begin{equation}
\oint_{\Sigma} \mathbf{E} \cdot \mathbf{\dx{S}} = \oint_{S_{+}} \mathbf{E} \cdot \mathbf{\dx{S}_{+}} + \oint_{S_{-}} \mathbf{E} \cdot \mathbf{\dx{S}}_{-} = E \left( \ihat \oint_{S_{+}} \dx{S}_{+} - (- \ihat) \oint_{S_{-}} \dx{S}_{-} \right) = E \cdot 2 (\pi r^2)
\end{equation} 
Dato che la densità di carica superficiale $\sigma$ è uniforme, si ha
\begin{equation}
Q_{T} = \sigma \pi r^2
\end{equation}
e dunque per la legge di Gauss
\begin{equation}
\dfrac{\sigma \pi r^2}{\ezero} = E \cdot 2 (\pi r^2)
\end{equation}
da cui, come nel precedente paragrafo, si ricava che
\begin{mdframed}
\begin{equation}
\mathbf{E} = \dfrac{\sigma}{2 \ezero} \versor{n}
\end{equation}
\end{mdframed}

\subsubsection{Distribuzione cilindrica uniforme indefinita}
Un cilindro \emph{indefinito} è un cilindro con raggio finito ed altezza infinita. Applicando il teorema di Gauss, calcoliamo il campo elettrostatico prodotto da carica distribuita con densità uniforme $\rho$ in un cilindro indefinito. Il sistema ha ovviamente simmetria cilindrica, per cui consideriamo superficie cilindrica $\Sigma$ di raggio di base $r$ e altezza (finita) $h$. Distinguiamo il caso in cui $r \leq R$ e quello in cui $r > R$.

Nel primo caso, la carica contenuta in $\Sigma$ è
\begin{equation}
Q_{I} = \pi r^{2} h \rho
\end{equation}
Per la simmetria del sistema, il campo deve essere radiale e uniforme su tutta la superficie laterale $\Sigma_{L}$ del cilindro. Essendo radiale, esso è parallelo alle due basi, e ha quindi flusso nullo attraverso esse. Essendo uniforme sulla superficie laterale, esso si può raccogliere fuori dall'integrale di superficie. Si ha quindi che il flusso attraverso $\Sigma$ è 
\begin{equation}
\oint_{\Sigma} \mathbf{E} \cdot \mathbf{\dx{S}} = \oint_{\Sigma_{L}} \mathbf{E} \cdot \mathbf{\dx{S}} = E \oint_{\Sigma_{L}} \dx{S} = E \cdot 2 \pi r h = \dfrac{\pi r^{2} h \rho}{\ezero}
\end{equation}
dove l'ultima uguaglianza è la legge di Gauss. Si ricava quindi 
\begin{equation}
\mathbf{E}(r \leq R) = \dfrac{\rho}{2 \ezero} r \rhat %la notazione non è esattamente rigorosa, vabbè
\end{equation}
Il secondo caso è analogo, con l'eccezione che ora la carica contenuta in $\Sigma$ non cambia con $r$:
\begin{equation}
Q_{I} = \pi R^{2} h \rho
\end{equation}
e procedendo come sopra si ottiene che
\begin{equation}
\mathbf{E}(r > R) = \dfrac{\rho R^{2}}{2 \ezero} \dfrac{1}{r} \rhat
\end{equation}
Riassumendo, quindi, si ha
\begin{mdframed}
\begin{equation}
\mathbf{E} = \left\lbrace \begin{matrix}
\dfrac{\rho}{2 \ezero} r \rhat \quad \mathrm{se\ } r \leq R \\
\\
\dfrac{\rho R^{2}}{2 \ezero} \dfrac{1}{r} \rhat \quad \mathrm{se\ } r > R \\
\end{matrix} \right.
\end{equation}
\end{mdframed}

\subsubsection{Guscio sferico uniformemente carico}
Questa distribuzione di carica corrisponde a quella che si ottiene caricando una sfera (piena) conduttrice. Per la simmetria del sistema, applichiamo il teorema di Gauss a superfici $\Sigma$ sferiche di raggio $r$. Anche qui distinguiamo il caso $r < R$  da $r > R$ (non abbiamo il caso di uguaglianza: il campo è discontinuo sulla superficie, dunque non ha un valore definito).
Nel primo caso, si nota subito che, poiché non c'è carica all'interno del guscio, deve essere
\begin{equation}
Q_{I} = 0
\end{equation}
e dunque, poiché per simmetria $\mathbf{E}$ deve essere uniforme in modulo su $\Sigma$ e diretto radialmente, e non possono quindi esserci cancellazioni del flusso, deve essere 
\begin{equation}
\mathbf{E}(r < R) = 0
\end{equation}
Nel secondo caso, invece, si ha che $Q_{I}$ è sempre la carica $Q$ dell'intero guscio. Quanto al flusso, invece, si ha
\begin{equation}
\oint_{\Sigma} \mathbf{E} \cdot \mathbf{\dx{S}} = E \oint_{\Sigma_{L}} \dx{S} = E \cdot 4 \pi r^{2} = \dfrac{Q}{\ezero}
\end{equation}
usando ancora una volta la legge di Gauss per l'ultima uguaglianza. Riorganizzando abbiamo
\begin{equation}
\mathbf{E}(r > R) = \dfrac{1}{4 \pi \ezero} \dfrac{Q}{r^2} \rhat
\end{equation}
Il campo esterno è quindi identico a quello prodotto da una carica uniforme posta nel centro del guscio. Riassumendo, abbiamo
\begin{mdframed}
\begin{equation}
\mathbf{E} = \left\lbrace \begin{matrix}
0 \quad \mathrm{se\ } r \leq R \\
\\
\dfrac{1}{4 \pi \ezero} \dfrac{Q}{r^2} \rhat \quad \mathrm{se\ } r > R \\
\end{matrix} \right.
\end{equation}
\end{mdframed}
Si noti che la discontinuità attraversando la superficie è in accordo con quella ricavata in precedenza. Integrando in $\dx{r}$ si ottiene il potenziale:
\begin{mdframed}
\begin{equation}
V = \left\lbrace \begin{matrix}
C \quad \mathrm{se\ } r \leq R \\
\\
- \dfrac{1}{4 \pi \ezero} \dfrac{Q}{r} + C \quad \mathrm{se\ } r > R \\
\end{matrix} \right.
\end{equation}
\end{mdframed}

\subsubsection{Spessore indefinito uniformemente carico}
Uno spessore indefinito è un prisma rettangolare con altezza e larghezza infinite e profondità $D$. Ponendo l'origine nel centro dello spessore e l'asse $x$ ortogonale ad esso, si consideri una superficie $\Sigma$ cilindrica con una base nel centro dello spessore\footnote{Per completezza, a lezione abbiamo usato un cilindro con il centro nel centro dello spessore e due basi che uscivano, basta moltiplicare tutto per 2.}, asse parallelo all'asse $x$, altezza $x$ e area di base $A$. Distinguiamo anche qui il calcolo del campo all'interno dello spessore ($-\frac{D}{2} \leq x \leq \frac{D}{2}$) da quello all'esterno ($x < -\frac{D}{2}$ o $x > \frac{D}{2}$). 
Nel primo caso, la carica contenuta in $\Sigma$ è 
\begin{equation}
Q_{I} = \rho A x
\end{equation}
Per la simmetria del sistema, il campo deve essere parallelo all'asse $x$ (quindi il suo flusso attraverso la superficie laterale del cilindro deve essere nullo) e uniforme sulle superfici parallele al piano. Inoltre, sempre per simmetria, il campo deve essere nullo nel centro dello spessore, per cui il suo flusso attraverso la base lì posizionata deve essere nullo. Si ha quindi
\begin{equation}
\oint_{\Sigma} \mathbf{E} \cdot \mathbf{\dx{S}} = E \oint_{\Sigma_{x}} \dx{S} = E \cdot A = \dfrac{\rho A x}{\ezero} 
\end{equation}
dove ancora una volta l'ultima uguaglianza è il teorema di Gauss. Per quanto detto sopra sulla simmetria planare del sistema, la forma vettoriale del campo all'interno dello spessore deve essere 
\begin{equation}
\mathbf{E} = \dfrac{\rho}{\ezero} x \ihat
\end{equation}
Nel caso in cui ci si trovi all'esterno, invece, la carica contenuta in $\Sigma$ vale 
\begin{equation}
Q_{I} = \rho A \dfrac{D}{2}
\end{equation}
e, con considerazioni analoghe a sopra, deve essere
\begin{equation}
\oint_{\Sigma} \mathbf{E} \cdot \mathbf{\dx{S}} = E \oint_{\Sigma_{x}} \dx{S} = E \cdot A = \dfrac{\rho A D}{2 \ezero} 
\end{equation}
e quindi
\begin{equation}
\mathbf{E} = \dfrac{\rho D}{ 2\ezero}  \ihat
\end{equation}
Dunque, riassumendo,
\begin{mdframed}
\begin{equation}
\mathbf{E} = \left\lbrace \begin{matrix}
\mathbf{E} = \dfrac{\rho}{\ezero} x \ihat \quad \mathrm{se\ } -\frac{D}{2} \leq x \leq \frac{D}{2}  \\
\\
\mathbf{E} = \dfrac{\rho D}{ 2\ezero}  \ihat \quad \mathrm{se\ } x < -\frac{D}{2} \mathrm{\ o\ } x > \frac{D}{2} \\
\end{matrix} \right.
\end{equation}
\end{mdframed}
Si noti che $\mathbf{E}$ cambia direzione spostandosi da una parte all'altra dell'origine. Inoltre, il campo elettrostatico della distribuzione è continuo ma per $D \rightarrow 0$ diventa discontinuo e coincidente con quello prodotto da un piano indefinito. %inserire potenziale anche interno?
\end{document}